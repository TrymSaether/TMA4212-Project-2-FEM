\documentclass[a4paper,10pt]{article}
\usepackage{trymtex}
% \usepackage[backend=biber,style=alphabetic]{biblatex}

\begin{document}
\begin{titlepage}
    \newcommand{\HRule}{\rule{\linewidth}{0.5mm}}
    \begin{tikzpicture}[remember picture, overlay]
      % NTNU logo
      \node[anchor=north west, xshift=1.0cm, yshift=-1.0cm] at (current page.north west) {
        \includegraphics[width=2.0cm]{figures/ntnu_logo_liten.png}
      };
    \end{tikzpicture}
  
    \center

    % Course code & title
    {\color{ntnu-blue}\sffamily\large TMA4212 \par}
    {\sffamily\Large Numerical Solution of Differential Equations by Difference Methods \par}
    
    \HRule
    \vspace{1.5cm}
  
    % Assignment title
    {\large\sffamily\bfseries Project 2\par}
    \vspace{0.3cm}
    {\Large\sffamily\textit{Solving the Poisson equation, and an Optimal Control Problem\\ using the Finite Element Method}\par}
  
    \vspace{0.5cm}
    \HRule
  
    \vfill
  
    % Author info
    \begin{minipage}{0.6\textwidth}
      \begin{flushleft}
        \large
        \textbf{Authors:}\\
        Haugen, Tor Ludvig Løvold \\
        Sæther, Trym\\ 
      \end{flushleft}
    \end{minipage}%
    \begin{minipage}{0.4\textwidth}
      \begin{flushright}
        \large
        \textbf{Semester:}\\
        Spring 2025
      \end{flushright}
    \end{minipage}
  
    % University logo/name
    \begin{center}
      {\color{ntnu-blue}\sffamily\Large Norwegian University of Science and Technology}\\
      \vspace{0.3cm}
      {\sffamily\large Department of Mathematical Sciences}
  
      \vspace{0.5cm}
      {\large\today}
    \end{center}
  
    \vspace{1cm}
  \end{titlepage}
  
  
  
\clearpage

\section{Poisson Equation}
\subsection{Mesh}
We define a grid $0 = x_0 < x_1 < \ldots < x_M = 1$ with $N$ elements $K_k = [x_i, x_{k+1}]$ for $k = 0, \ldots, N-1$.
The mesh size is then defined as $h_k = x_{k+1} - x_k$.
\subsection{Reference Element $\hat{K}$}
We define the quadratic shape functions $\psi_i \in \mathbb{P}_2$ on the reference element $\hat{K} = [0, 1]$, with $\xi_\beta \in \{0, \frac{1}{2}, 1\}$ as the reference nodes.
\begin{align*}
    \psi_0(\xi) &= \frac{(1-\xi)(\xi-\frac{1}{2})}{(-\frac{1}{2})(-1)} = 2\xi^2 - 2\xi + \frac{1}{2}\\
    \psi_1(\xi) &= \frac{(\xi)(\xi-1)}{(\frac{1}{2})(-\frac{1}{2})} = 4\xi(1-\xi)\\
    \psi_2(\xi) &= \frac{(\xi)(\xi-\frac{1}{2})}{(1)(\frac{1}{2})} = 2\xi(\xi-\frac{1}{2})
\end{align*}

Now we define the mapping from the reference element to the physical element $\Phi_K: \hat{K} \to K$ as:



\subsection{Basis Functions}
\begin{align*}
    u(x) &= \sum_i u_i \phi_i(x) = \mathbf{u}^T \symbf{\phi}(x)\\
    v(x) &= \sum_j v_j \phi_j(x) = \mathbf{v}^T \symbf{\phi}(x)\\
\end{align*}
Where $\phi_i(x)$ is the quadratic basis function defined as:
\begin{align*}
    \phi_i(x) &= \frac{(x-x_{i-1})(x-x_{i+1})}{(x_i-x_{i-1})(x_i-x_{i+1})}\\
    \phi_{i-1}(x) &= \frac{(x-x_i)(x-x_{i+1})}{(x_{i-1}-x_i)(x_{i-1}-x_{i+1})}\\
    \phi_{i+1}(x) &= \frac{(x-x_i)(x-x_{i-1})}{(x_{i+1}-x_i)(x_{i+1}-x_{i-1})}
\end{align*}







\end{document}